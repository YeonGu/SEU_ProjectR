\chapter{系统中断}

在这一章节,我们将会学到嵌入式中最重要的事件处理机制——中断机制,它能够实现系统异步事件处理并提高系统效率

\section{什么是中断?}
什么叫做中断?这要从处理器的工作方式说起。在嵌入式系统中,处理器不会专门等待某一个特定事件发生然后再执行对应的操作,因为这样会浪费大量的资源(当事件没有发生的时候,处理器就白白等待了)。与此相反,它会保持不断工作或者处于休眠的状态。当特定事件发生时,相关外设会给处理器发送处理该事件的信号,处理器收到信号后,会暂时停下手中的活,先处理特定事件,完成后再回过来继续做之前没做完的事情。这就像是特定事件打断(Interrupt)了处理器的工作,因此这种工作机制被称为中断机制,这样的特定事件被称为中断事件。而中断事件发生,并且给处理器传递了相关信号的过程,就叫做产生了中断。

和中断相反的机制叫做轮询(Polling)。在轮询机制中,系统会循环地检查特定的状态或条件,来判断是否要进行相对应的操作。然而在嵌入式系统这种极致追求资源利用的地方,如果没有检测到任何事件,检测过程中所消耗的资源就被浪费掉了。因此一般不采用轮询的机制。

根据触发方式的不同,中断又可以分为上升沿中断和
\section{执行流的切换}

处理器响应并处理中断事件实际上就是执行流的切换。我们不妨捋一捋中断是怎么一步步被处理器响应并处理的。
\begin{enumerate}
    \item {\bf 外部事件触发:}外部事件可以来自硬件设备(比如定时器到期,引脚状态发生变化),也可以来自软件(比如软件中断请求)。
    \item {\bf 中断请求发起:}当检测到事件发生后,外设会向处理器发出中断请求信号,表示需要处理器立刻响应并处理相应的事件。
    \item {\bf 中断检测和响应:}处理器在执行当前任务的过程中,会周期性地检测中断请求信号。当处理器检测到外设发来的信号之后,就会暂停当前任务的执行,保存当前任务的执行状态,并开始执行对应的中断处理程序。
    \item {\bf 中断服务例程执行:}中断服务例程就是中断处理程序,决定了处理器应该怎么处理对应的中断事件。
    \item {\bf 中断结束:}中断服务例程结束后,处理器会返回到原先被打断的任务的执行点,继续执行原来的任务,直到下一次中断产生。
\end{enumerate}
着重看一看处理器在收到中断信号之后是怎么开始执行中断服务例程的。
\begin{enumerate}
    \item {\bf 保存当前执行状态:}比如程序计数器,堆栈指针,寄存器状态等等关键信息。
    \item {\bf 选择新的执行流:}处理器找到对应的中断服务例程。
    \item {\bf 恢复新的执行状态:}处理器会根据新执行流的信息,设置好程序计数器,堆栈指针,寄存器等等,准备进入新的执行流。
    \item {\bf 执行新的指令流:}处理器开始新的执行流,执行相应的中断服务例程中的指令。
\end{enumerate}
通过以上步骤,处理器就实现了从原先的执行流跳转到新执行流的完整过程。这就称为“执行流的切换”。当中断服务流程结束后,处理器再次切换执行流,回到原先的任务中。
\section{处理器去哪找中断服务例程?——中断向量表}
在上面切换执行流的过程中,最重要的就是该切换到哪一个执行流?处理器去哪里找一个中断事件对应的中断服务例程?这就需要中断向量表了。

首先什么是向量表?向量可以表示方向,指向想去的地方。在计算机中,向量可以用地址值来表示,这样就能指明方向,该去哪找某个值或函数等等。向量表就是一组地址值组成的集合,它把某一类值或者函数的地址集中放在一起。

而中断向量表储存的就是中断服务例程的地址。在STM32的启动文件中有这样的汇编代码:

\begin{lstlisting}
; Vector Table Mapped to Address 0 at Reset
                AREA    RESET, DATA, READONLY
                EXPORT  __Vectors
                EXPORT  __Vectors_End
                EXPORT  __Vectors_Size

__Vectors       DCD     __initial_sp               ; Top of Stack
                DCD     Reset_Handler              ; Reset Handler
                DCD     NMI_Handler                ; NMI Handler
; 中间部分省略
__Vectors_End
\end{lstlisting}

我们不妨试试手撕汇编,看看这部分到底在干啥。首先\verb|AREA|指令定义了一个名为“\verb|RESET|”的区域,并告诉编译器这部分区域用来储存数据(\verb|DATA|),而且是只读的(\verb|READONLY|)。
接着导出了三个值(\verb|EXPORT|),以便在外部调用,这三个值会在接下来看到。然后开始向量表的主要部分。

指令\verb|DCD|的作用是定义一个32位的常量数据,并分配储存空间储存这个数据。首先定义并分配了两个字节储存一个名为\verb|__initial_sp|
的值(这个名称代表了栈顶指针的地址);接着继续\verb|DCD|,定义并分配了两个字节储存\verb|Reset_Handler|代表的值(这是复位中断服务例程的地址);再继续\verb|DCD|,定义并分配了两个字节储存\verb|NMI_Handler|的值(这是不可屏蔽中断服务例程的地址)。
依次类推,不断\verb|DCD|下去,不断地分配空间储存了各个中断服务例程的地址。请注意,连续\verb|DCD|会连续分配空间,因此这一列\verb|DCD|下来,我们就得到了一块连续的空间,连续地储存了各个中断服务例程的地址,这块空间就是中断向量表。

为了方便我们表示或调用这块空间的地址,我们在第一次\verb|DCD|的前面加上\verb|__Vectors|,用它来标记中断向量表的起始地址。同理,我们在最后用\verb|__Vectors_End|来表示向量表的结束地址。而两个地址相减,就得到了向量表的大小,用\verb|__Vectors_Size|表示(它的定义在向量表定义的下面)。
\begin{lstlisting}
    __Vectors_Size  EQU  __Vectors_End - __Vectors
\end{lstlisting}
其中汇编指令\verb|EQU|用来定义符号常量,也就是给在\verb|EQU|后面的值起一个名称,而名称就是\verb|EQU|前面的表达式。

于是我们知道了,中断向量表在启动文件中被定义了,它是一块连续的区域,储存了每个中断服务例程的地址。当处理器需要处理中断事件时,只需来这里这里找对应服务例程的地址,然后直接跳转到对应的位置开始执行指令流即可。
\section{中断服务例程会做些什么?}
找到并跳转至中断服务例程后,接下来的问题是,中断服务例程会做些什么?答案是,现在它会让系统陷入死循环。所有的中断服务例程被统一放在\verb|stm32f4xx_it.h/.c|两个文件里,前者存放函数接口和相关的宏定义,后者存放函数的实现。打开源文件,可以发现所有的中断服务例程都是这样的
(\verb|ppp|是中断的名称)
\begin{lstlisting}
void ppp_Handler(void)
{
    while (1) {}
}
\end{lstlisting}
当中断发生后,处理器跳转到对应函数\verb|ppp_Handler|所在地址,然后执行这个函数。但在函数里只有一个\verb|while(1)|,因此处理器将陷入死循环。而因为所有的中断函数都是这样的死循环,所以只要发生了中断事件,处理器将永远无法返回先前的执行流,直到系统被复位。

显然这不是我们想要的结果,我们希望处理器处理完中断事件后回到原先的执行流。事实上,\verb|stm32f4xx_it.c|

\noindent 仅仅是给你提供了一个模板。中断服务例程要干什么事情,是需要自己编写的。这里全部写成死循环,是为了防止某些未编写服务例程的中断事件发生,进而导致系统异常。

至此,我们已经知道了当中断事件发生后,处理器会去哪找中断服务例程,中断服务例程会做些什么。最重要的是,我们知道了该去哪个文件编写满足我们自己需要的中断服务例程。接下来我们将上手试试看。

\section{不一样的点灯方法——GPIO中断}
在STM32中,几乎每一个外设都能产生中断,我们先从最基础的GPIO中断开始,研究如何利用中断机制编程。

\subsection{实验操作}
为了使用GPIO中断,还需要补充EXTI(External Interrupt)外部中断控制器的知识。请观看视频\href{【【keysking的stm32教程】 第6集 狂飙STM32中断】 https://www.bilibili.com/video/BV1Fj411V7aq/?share_source=copy_web&vd_source=00b9d329964a93c9843f9c524074f948}{初识中断}和\href{https://www.bilibili.com/video/BV1M24y1473t/?share_source=copy_web&vd_source=00b9d329964a93c9843f9c524074f948}{深入中断},
尝试自己实现GPIO中断点灯,并解释一遍GPIO中断信号进入到中断控制器的过程。

\section{RTFSC(1)}
\subsection{中断标志位}
跳转到CubeMX自动帮我们生成的\verb|HAL_GPIO_EXTI_IRQHandler|函数的定义
\begin{lstlisting}
void HAL_GPIO_EXTI_IRQHandler(uint16_t GPIO_Pin)
{
  /* EXTI line interrupt detected */
  if(__HAL_GPIO_EXTI_GET_IT(GPIO_Pin) != RESET) // 检测中断标志位
  {
    __HAL_GPIO_EXTI_CLEAR_IT(GPIO_Pin); // 清除中断标志位
    HAL_GPIO_EXTI_Callback(GPIO_Pin); // 回调函数
  }
}
\end{lstlisting}
各种中断发生时,都会提醒处理器这里有一个中断需要处理。提醒的方法就是设置一个对应的中断标志位,并将其置1。当处理器处理完中断后,就要将标志位置0,否则会一直提醒处理器这里有中断需要被处理。有一些外设能够在中断服务例程开始时自动清除中断标志位(硬件设计),但
遗憾的是EXTI没有这样的设计,只能手动清除中断标志位。我们不妨试试不清除中断标志位。将清除中断标志位的代码注释掉,然后再编译并烧录。我们会发现如果一直按着按键,小灯将会不停闪烁,说明中断服务例程正在不断被触发。
试试自行理顺逻辑,并讲给自己听,为什么会发生这种情况?

\subsection{关于检测中断信号的问题}
回到主文件\verb|main.c|,主函数是这样的
\begin{lstlisting}
int main(void) 
{
    HAL_Init();
    SystemClock_Config();
    MX_GPIO_Init();

    while (1)
    {
        /* 内容省略 */
    }
}
\end{lstlisting}
你也许会有些疑问,我们说处理器是要检测到中断信号才会响应并处理中断,那它不是要一直检测中断信号吗?那检测的代码不是应该放在\verb|while|循环里反复执行吗?但我们写的循环里明显没有检测中断信号的代码,那检测中断的代码哪去了呢?
这要从处理器执行指令的方式说起。简单来讲,处理器执行指令的流水线是:从内存读取指令->解释指令->执行指令->{\bf 检测中断}。处理器不是单纯翻译并执行指令,
还会在每个指令周期的最后检测一次中断。也就是说,处理器的工作方式决定了检测中断这事,就是自动进行的,不需要你自己编写。回到主程序中的代码,处理器每执行一条指令
都会自动附加一次中断检查。主循环不断循环,就能不断地检测中断。因此,我们只需要初始化好相应外设,就能静静等待中断被触发并被处理器响应,进而执行中断服务例程(反转灯的亮灭)。

当然,即使现在你已经知道了处理器确实在自动检测中断,但可能还有一个问题,处理器是怎么检测中断的呢?在STM32中,各个外设都会将中断信息传到NVIC中断控制器处(这是通过电路设计实现的)。
处理器只需扫描NVIC中断控制器就能获得中断信息。而NVIC就是接下来将要讲到的内容。

\section{NVIC嵌套中断向量控制器}
NVIC(Nested Vectored Interrupt Controller),译为嵌套中断向量控制器。它与内核紧密耦合,属于内核里面的一个外设,是ARM Cortex-M微控制器用于管理中断的核心组件。除了上面实验中见到的使能中断,它还能失能中断,控制中断优先级,处理中断服务例程的嵌套等等。

常见的与NVIC相关的库函数有
\begin{lstlisting}
void HAL_NVIC_EnableIRQ(IRQn_Type IRQn); // 使能中断
void HAL_NVIC_DisableIRQ(IRQn_Type IRQn); // 失能中断
void HAL_NVIC_SetPendingIRQ(IRQn_Type IRQn); // 设置中断标志位
void HAL_NVIC_ClearPendingIRQ(IRQn_Type IRQn); // 清除中断标志位
void HAL_NVIC_SetPriority(IRQn_Type IRQn, uint32_t PreemptPriority, uint32_t SubPriority); // 设置中断优先级
\end{lstlisting}
前两个函数我们已经见过了,一部分模块产生中断的功能默认情况下是禁用(失能)的,想要使用需要先使能,如果不想再使用则可以将其失能。而第三个和第四个函数是用来检测和清除中断标志位的,而在上面的实验里我们用另两个函数实现了同样的功能。感兴趣的话可以试试用这两个函数再实现一次上面的功能。

我们的重点在于最后一个函数,中断优先级的设置。
\subsection{中断优先级}
在GPIO中断里,我们只需要用到一种中断。但如果需要用到多种中断,或者有多个中断事件同时发生会怎样呢?显然处理器是不能同时处理多个中断事件的。

为了解决这个问题,我们引入中断优先级的概念。当多个中断事件同时发生时,处理器会根据中断事件的重要性从高到低,按顺序处理这些事件。这个重要性排序,就是中断优先级,越重要的中断事件优先级就越高。
当然你可以问,你怎么知道对我来讲哪些事件更重要?的确,不同情况下对优先级的要求是不一样的,正因此,HAL库为我们提供了设置中断优先级的函数\verb|HAL_NVIC_SetPriority|,除了一些不可变的中断事件(比如总线错误,内核错误等),其他的外设中断事件的优先级都是可以调整的。

但这个函数需要提供三个参数,其中\verb|IRQn|是中断号,另外两个是什么呢?

在NVIC里有一个专门的寄存器:中断优先级寄存器\verb|NVIC_IPRx|用来配置外部中断的优先级。在Cortex-M4内核中,该寄存器的定义如下
\begin{lstlisting}
volatile uint8_t  IP[240U];
\end{lstlisting}
数组中的每个元素都是8位无符号整型,储存一个中断的优先级信息,因此理论上最多可以储存240个中断的优先级。但CM4芯片做了精简设计,导致每个8位无符号整型实际只有最高的四位用来表达优先级,最低的四位则保留。
而用来表达优先级的这四位,又被分为抢占优先级(PreemptPriority)和子优先级(SubPriority)。如果有多个中断同时响应,抢占优先级高的中断就会比抢占优先级低的中断优先被处理,如果抢占优先级相同,就比较子优先级。
如果子优先级还一样,则比较中断号,中断号越小,优先级越高。

在这表达优先级的四位里,有几位表示抢占优先级,剩下几位表示子优先级,是可以自行设置的。根据分配位数的不同,可以给优先级分组编号
\begin{lstlisting}
NVIC_PriorityGroup_0 // 0bit抢占优先级 4bit子优先级
NVIC_PriorityGroup_1 // 1bit抢占优先级 3bit子优先级
NVIC_PriorityGroup_2 // 2bit抢占优先级 2bit子优先级
NVIC_PriorityGroup_3 // 3bit抢占优先级 1bit子优先级
NVIC_PriorityGroup_4 // 4bit抢占优先级 0bit子优先级

void NVIC_PriorityGroup(uint32_t PriorityGroup) {
    NVIC_SetPriorityGrouping(PriorityGroup);
}
\end{lstlisting}
上面展示了每个分组编号对应的分配情况,以及HAL库中设置优先级分组的函数。这里抢占优先级有\verb|n|位,是从最高位向下数\verb|n|位;子优先级有\verb|n|位,是从最低位向上数\verb|n|位。

默认情况下(即不做任何优先级设置),各个中断的优先级是按照中断号排序的,中断号越小,优先级越高,但一般情况下我们都会根据所需调整优先级顺序。
\subsection{关于中断号}
在\verb|stm32f411xx.h|中有名为\verb|IRQn_Type|的枚举类型,里面枚举了所有中断类型对应的中断号
\begin{lstlisting}
typedef enum
{
    NonMaskableInt_IRQn         = -14,
    /* 中间部分省略 */
    SysTick_IRQn                = -1,

    WWDG_IRQn                   = 0,
    /* 中间部分省略 */
    FPU_IRQn                    = 81
} IRQn_type;
\end{lstlisting}
中断类型可以分为两种,第一种的中断号小于0,表示中断类型是内核中断;第二种的中断号大于0,表示中断类型是外设中断。内核中断的优先级是不可编程的(从对应的中断事件也可以看得出来它们的重要性的确很高),
而外设中断的优先级则可以根据需要改变。

在\verb|NVIC_IPRx|里,每个中断的优先级信息根据其中断号储存在\verb|IP|数组中对应的位置上。比如中断号为6的EXTI0中断,就储存在\verb|IP[6]|处。
对于中断号小于零的中断类型,不妨在\verb|__NVIC_SetPriority|函数的定义中探索一下对应关系。

\section{让小灯定时闪烁——定时器中断}
这一节我们将尝试另一种中断——定时器中断。定时器(Timer)是实现定时功能最基础的外设,在STM32F4xx系列中定时器分为几种,但这里只需要用到基本定时器,其余则留到相应章节讲解。
\subsection{定时器原理}
请观看视频\href{https://www.bilibili.com/video/BV11u4y1A7gS/?share_source=copy_web&vd_source=00b9d329964a93c9843f9c524074f948}{基本定时功能},尝试自行复述一遍定时器的计时原理,并在CubeMX中正确配置基本定时器。
与串口相关的内容可自行略过。
\subsection{代码编写}
我们尝试利用定时器中断实现小灯的定时闪烁。首先在CubeMX中设置好预分频和自动重装载值,使得产生中断的间隔为1s,并开启对应的NVIC中断。生成代码后,首先要在主函数中开启基本定时器中断
\begin{lstlisting}
HAL_TIM_Base_Start_IT(&htim6);
\end{lstlisting}
接着编写中断服务例程。
\begin{lstlisting}
void TIM6_DAC_IRQHandler(void)
{
    HAL_TIM_IRQHandler(&htim6); // 已自动生成
}

void HAL_TIM_PeriodElapsedCallback(TIM_HandleTypeDef *htim) // 计数到达自动重装载值时的回调函数
{
    if (htim == (&htim6)) // 检测是否是TIM6到达计数周期
    {
        HAL_GPIO_TogglePin(LD1_GPIO_Port, LD1_Pin); // 反转灯的亮灭
    }
}
\end{lstlisting}
最后主函数中应该会是这个样子的
\begin{lstlisting}
int main(void) {
    HAL_Init();
    SystemClock_Config();
    MX_GPIO_Init();
    MX_TIM6_Init();

    HAL_TIM_Base_Start_IT(&htim6);

    while (1) {}
}
\end{lstlisting}
编译并烧录,可以发现单片机上的LED灯以1s的间隔闪烁。
\section{RTFSC(2)}
这里的中断服务例程和GPIO中断不同,除了\verb|TIM6_DAC_IRQHandler|以外,还使用了另一个回调函数,这是为什么呢?
事实上,定时器中断可以由很多种中断事件触发,显然需要不一样的处理方法,自然也就需要不同的函数。这里我们用到的
\verb|PeriodElapsedCallback|函数,就是专门用来处理计时器更新事件的。

我们不妨跳转到\verb|HAL_TIM_IRQHandler|的定义。这个函数中有很多判断条件,简单来讲,就是根据计数器状态寄存器中的信息,依次判断产生定时器
中断的原因(感兴趣的话可以对照参考手册中关于定时器寄存器的说明,研究一下这些判断语句是怎么读取寄存器中的信息的)。在这个实验中,当定时器中断产生时
,处理器进入\verb|TIM6_DAC_IRQHandler|函数,接着进入\verb|HAL_TIM_IRQHandler|函数,逐个判断中断事件类型;判断发生的事件是计时器更新事件后,
进入\verb|PeriodElapsedCallback|函数,执行反转灯的亮灭的指令。
\begin{lstlisting}
if ((itflag & (TIM_FLAG_UPDATE)) == (TIM_FLAG_UPDATE))
{
    if ((itsource & (TIM_IT_UPDATE)) == (TIM_IT_UPDATE)) // 判断事件是否为计时器更新事件
    {
      __HAL_TIM_CLEAR_FLAG(htim, TIM_FLAG_UPDATE); // 清除计时器更新事件中断标志位
#if (USE_HAL_TIM_REGISTER_CALLBACKS == 1) // 开启手动注册回调函数机制(不做说明)
      htim->PeriodElapsedCallback(htim); 
#else
      HAL_TIM_PeriodElapsedCallback(htim); // 调用处理计时器更新事件的函数
#endif /* USE_HAL_TIM_REGISTER_CALLBACKS */
    }
  }
\end{lstlisting}
在\verb|HAL_TIM_IRQHandler|中还有大量的条件编译,但我们在实验中显然没有操作过任何操作过任何一个编译条件,因此暂时可以不管,只要看\verb|#else|后面的语句就好。
有条件的话也可以研究一下,这个编译条件是什么意思,开启之后会发生什么。

\section{用键盘控制小灯——UART中断}
在这个实验中,我们将利用前面讲到的UART串口通讯,让小灯“按着我们的意思”来闪烁。请观看视频
\href{https://www.bilibili.com/video/BV1bc411J7Tv/?p=10&share_source=copy_web&vd_source=00b9d329964a93c9843f9c524074f948}{串口中断},尝试利用串口通讯实现
通过在键盘上输入不同数字,实现改变灯的亮灭。

\section{缓冲区与消息队列}
在中断机制中,缓冲区和消息队列是两种常用的数据结构,它们用于管理和存储中断相关的数据。
\subsection{缓冲区}
缓冲区是内存中的一块区域,用于临时存储数据。在中断上下文中,缓冲区通常用于存储从硬件设备(如储存介质,键盘等)传入的数据,或者存储准备发送到硬件设备的数据。当硬件设备准备好接收或发送数据时,它会发出中断信号,操作系统将数据从设备的缓冲区复制到内核的缓冲区,或者从内核的缓冲区复制到设备的缓冲区。

一种特殊的缓冲区叫环形缓冲区,它在数据结构上首尾相连,形成一个环状,因此得名(实际储存在内存中当然还是线性的)。环形缓冲区环形缓冲区在创建时被分配为一个固定大小的数组,即它有一定的容量限制。它用两个指针来跟踪数据的写入和读取位置。写指针指示下一个数据应该被写入的位置,而读指针指示下一个数据应该被读取的位置。
如果写指针追上读指针,新的数据将覆盖旧的数据,或者也可以定义一些其他的处理策略。但普通的线性缓冲区写满后就无法再写入了。
相比普通线性缓冲区,环形缓冲区一个很大的优势就是数据被写入和读取时,不需要移动其他数据,大大减少了内存操作的次数。
\subsection{消息队列}
消息队列是一种数据结构,它允许消息按顺序存储和检索。在中断机制中,消息队列用于管理中断请求和中断处理程序的通信。当一个中断发生时,它可以向消息队列中添加一个消息,该消息包含了中断的详细信息。操作系统或中断服务例程可以按顺序处理消息队列中的消息,确保每个中断都得到适当的响应。

消息队列的好处是它提供了一种有序的方式来处理多个中断,可以确保高优先级的中断得到优先处理,并且不会因为低优先级的中断而延迟。

